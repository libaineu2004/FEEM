\section{链接库的原理及使用}
\subsection{动态链接库}
\subsection{静态链接库}
\subsection{如何识别lib文件为静态链接库还是导入库?}
\section{Python脚本与C++程序的结合}

\section{Qt}
\subsection{Q\_DECLARE\_METATYPE}
这个宏是为了让QMetaType知道Type这个数据类型,并提供一个默认的拷贝构造函数和析构函数。QVariant需要使用Q\_DECLARE\_METATYPE这个宏来定制类型。

当使用这个宏的时候要求Type是一个完整的数据类型。可以使用Q\_DECLARE\_OPAQUE\_POINTER()来注册一个指针用于转发声明类型。

一般都把这个宏放到结构体或类的末尾【注意:这是官方说的】,如果不放到末尾也是阔以的,就放到头文件中,当你用 QVariant就要包含那个.h,个人觉得这非常不适合面向对象以及模块化编程。

通过添加Q\_DECLARE\_METATYPE()这个宏让QOject及其子类知道这个类型。这里要注意的是如果要在队列信号(关于connect函数队列信号请看这篇博文:https://blog.csdn.net/qq78442761/article/details/81937837)中使用或者用用槽连接,要先调用这个函数qRegisterMetaType()【这里是在运行的时候,对他进行注册】

\section{C++}
\subsection{左值、左值引用、右值、右值引用}
1、左值和右值的概念

左值是可以放在赋值号左边可以被赋值的值;

左值必须要在内存中有实体;

右值当在赋值号右边取出值赋给其他变量的值;

右值可以在内存也可以在CPU寄存器。

一个对象被用作右值时,使用的是它的内容(值),被当作左值时,使用的是它的地址。

2、引用

引用是C++语法做的优化,引用的本质还是靠指针来实现的。引用相当于变量的别名。

引用可以改变指针的指向,还可以改变指针所指向的值。

引用的基本规则:

声明引用的时候必须初始化,且一旦绑定,不可把引用绑定到其他对象;即引用必须初始化,不能对引用重定义;
对引用的一切操作,就相当于对原对象的操作。

3、左值引用和右值引用

3.1 左值引用

左值引用的基本语法:type \&引用名 = 左值表达式;

3.2 右值引用

右值引用的基本语法type \&\&引用名 = 右值表达式;

右值引用在企业开发人员在代码优化方面会经常用到。

右值引用的“\&\&”中间不可以有空格。