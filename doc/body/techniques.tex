\section{链接库的原理及使用}
\subsection{动态链接库}
\subsection{静态链接库}
\subsection{如何识别lib文件为静态链接库还是导入库?}
\section{Python脚本与C++程序的结合}

\section{Qt}
\subsection{Q\_DECLARE\_METATYPE}
这个宏是为了让QMetaType知道Type这个数据类型,并提供一个默认的拷贝构造函数和析构函数。QVariant需要使用Q\_DECLARE\_METATYPE这个宏来定制类型。

当使用这个宏的时候要求Type是一个完整的数据类型。可以使用Q\_DECLARE\_OPAQUE\_POINTER()来注册一个指针用于转发声明类型。

一般都把这个宏放到结构体或类的末尾【注意:这是官方说的】,如果不放到末尾也是阔以的,就放到头文件中,当你用 QVariant就要包含那个.h,个人觉得这非常不适合面向对象以及模块化编程。

通过添加Q\_DECLARE\_METATYPE()这个宏让QOject及其子类知道这个类型。这里要注意的是如果要在队列信号(关于connect函数队列信号请看这篇博文:https://blog.csdn.net/qq78442761/article/details/81937837)中使用或者用用槽连接,要先调用这个函数qRegisterMetaType()【这里是在运行的时候,对他进行注册】