\parindent0pt
\null
\thispagestyle{empty}
%\vskip3cm
\vfill
\hfil
%: définition des couleurs et des styles
% fond Gauche et Droite bleu marron
\definecolor{couleurGauche}{rgb}{.25,1,1}
\definecolor{couleurDroite}{rgb}{.75,.5,0.25}
% Post-It jaune
\definecolor{couleurPostIt}{rgb}{.9,.9,.35}
% logo TikZ dégradé blanc gris Top et Bottom
\definecolor{couleurTop}{rgb}{1,1,1}
\definecolor{couleurBottom}{rgb}{.5,.5,.5}



\tikzset{fondA/.style={ball color=couleurGauche}}
\tikzset{fondB/.style={ball color=couleurDroite}}
\tikzset{fondPostit/.style={color= couleurPostIt}}
\tikzset{ombrePunaise/.style={color={blue!10!gray}}}
\tikzset{ombrePostit/.style={color={black},opacity=.5}}
\tikzset{punaise/.style={ball color=red}}

% ============ utilitaires de construction ============
%: \ruban{angle}{point}
\newcommand{\ruban}[2]{\shadedraw[rotate=#1] #2
  ++(0:0.57735*\arete-0.57735*\epaisseur+2*\rayon)
  ++(-30:\epaisseur-1.73205*\rayon)
  arc (60:0:\rayon)   -- ++(90:\epaisseur)
  arc (0:60:\rayon)   -- ++(150:\arete)
  arc (60:120:\rayon) -- ++(210:\epaisseur)
  arc (120:60:\rayon) -- cycle;}

%: \logoTikZ{(point)}{angle}{TikZ}{pour}{L'impatient}
\newcommand{\logoTikZ}[5]
{\def\arete{3}   \def\epaisseur{5}   \def\rayon{2}
\begin{scope} [very thick,top color= couleurTop,bottom color= couleurBottom,rotate=#2]
	\coordinate (a) at ($#1+(0,-6)$);
	\coordinate (b) at ($#1+(6,-6)$);
	\coordinate (c) at ($#1+(6,6)$);
	\coordinate (d) at ($#1+(0,6)$);
	\ruban{0}{#1};
	\ruban{120}{#1};
	\ruban{-120}{#1};
	\draw #1 ++(-60:3.5) node[scale=5,rotate={#2+30}]{#3};
	\draw #1 ++(180:3.5) node[scale=3,rotate={#2-90}]{#5};
	\clip (a) -- (b) -- (c) -- (d) -- cycle; % pour croiser
	\ruban{0} {#1};
	\draw #1 ++(60:3.5) node [gray,xscale=-3,yscale=3,rotate={360-#2+30}]{#4};
\end{scope}}

% un trait épais et trois ovales empilés
%: \epingle{point}{angle}{échelle}
\newcommand{\epingle}[3]{
\coordinate[rotate=#2,yshift={#3*0.375cm}] (e) at #1;
\coordinate[shift={++(60:0.75)}] (g) at (e);
\begin{scope} [scale=1.5]
 \begin{scope}[rotate=-30]
   \coordinate[shift={++(30:0.75)}] (h) at (e);
   \draw[ombrePunaise,line cap=round,line width=4pt] (e) -- ++(60:0.75);
   \fill [ombrePunaise,rotate=-30,scale=0.5] (h) ellipse (.65 and .3) ;
   \fill [ombrePunaise,rotate=60,scale=0.5] (h) ++(0.4,0) ellipse (.4 and .3);
   \fill [ombrePunaise,rotate=60,scale=0.5] (h) ++(0.8,0) ellipse (.2 and .4);
 \end{scope}
 \draw[line cap=round,line width=4pt] (e) -- ++(60:0.75);
 \fill [punaise,rotate=-30,scale=0.5] (g) ellipse (.65cm and .3cm) ;
 \fill [punaise,rotate=60,scale=0.5] (g) ++(0.4,0) ellipse (.4 and .3);
 \fill [punaise,rotate=60,scale=0.5] (g) ++(0.8,0) ellipse (.2 and .4);
\end{scope}}


%: \postIt{(point)}{angle}{échelle}{ligne 1}{ligne 2}{ligne 3}}
\newcommand{\postIt}[6]
{\begin{scope} [rotate=#2]
\fill [red,ombrePostit] #1 ++ ($#3*(-1.45,0.72)$) -- ++ ($#3*(2.86,0)$)
 .. controls+(0,0)and+($#3*(-0.25,0.05)$).. ++ ($#3*(0.25,-2.4)$)
 .. controls+($#3*(-0.1,-0.1)$)and+(0,0).. ++ ($#3*(-2.95,0.1)$)
 -- cycle;
\fill [ombrePostit] #1 ++ ($#3*(-1.45,0.72)$) -- ++ ($#3*(2.86,0)$)
 .. controls+(0,0)and+($#3*(-0.25,0.05)$).. ++ ($#3*(0.2,-2.35)$)
 .. controls+($#3*(-0.1,-0.1)$)and+(0,0).. ++ ($#3*(-2.95,0.1)$)
 -- cycle;
\fill [ombrePostit] #1 ++ ($#3*(-1.45,0.72)$) -- ++ ($#3*(2.86,0)$)
 .. controls+(0,0)and+($#3*(-0.25,0.05)$).. ++ ($#3*(0.15,-2.3)$)
 .. controls+($#3*(-0.1,-0.1)$)and+(0,0).. ++ ($#3*(-2.95,0.1)$)
 -- cycle;
\fill [fondPostit] #1 ++ ($#3*(-1.45,0.72)$) -- ++ ($#3*(2.86,0)$)
 .. controls+(0,0)and+($#3*(-0.2,0.1)$).. ++($#3*(0.1,-2.25)$)
 .. controls+($#3*(-0.1,-0.1)$)and+(0,0).. ++ ($#3*(-2.95,0.1)$)
 -- cycle;
\end{scope}
\epingle{#1}{#2}{#3}
% texte
\draw #1 node [scale=#3,rotate=#2] {#4};
\draw #1 node [scale=#3,rotate=#2,below={#3*0.05cm}] {#5};
\draw #1 node [scale=#3,rotate=#2,below={#3*0.2cm}] {#6};}

% pour un placement précis des divers éléments
\tikzset{ajustage/.style={yshift=0.cm}}

%: ===== PAGE

%: XeLaTeX \font\fantaisie pour sous-titre et texte du Post-It
\font\fantaisie="Lucida Handwriting" at 10pt

\thispagestyle{empty}
%===========================================================================
\begin{tikzpicture}[remember picture,overlay]
% fond bicolore gauche droite
\coordinate [ajustage] (cp) at (current page);
\coordinate [ajustage] (cpe) at (current page.east);
\coordinate [ajustage] (cpne) at (current page.north east);
\coordinate [ajustage] (cpn) at (current page.north);
\coordinate [ajustage] (cpnw) at (current page.north west);
\coordinate [ajustage] (cpw) at (current page.west);
\coordinate [ajustage] (cpsw) at (current page.south west);
\coordinate [ajustage] (cps) at (current page.south);
\coordinate [ajustage] (cpse) at (current page.south east);
\fill[fondA] (cps) .. controls (cpw) and (cpe) .. (cpn) -- (cpnw)  -- (cpsw) -- cycle;
\fill[fondB] (cps) .. controls (cpw) and (cpe) .. (cpn) -- (cpne)  -- (cpse) -- cycle;
\draw[white,line width=4pt] (cps) .. controls (cpw) and (cpe) .. (cpn);
\draw[white,line width=8pt] (cpnw)--(cpne)--(cpse)--(cpsw)--cycle;


% Logo à peu en dessous du centre (bord interne aligné avec la courbe)
\logoTikZ{(cp)}{25}{\scriptsize Notebook}{p{\color{orange}oo}{\color{gray}f}{\color{orange}ee}}{\Large FEM}

% TikZ pour l'impatient en haut à gauche
\draw (cp)  node [xshift=-9.5cm,yshift=12cm,scale=6.5,right] {有{\color{orange}\textit{限元}}};
\draw (cp)  node [xshift=-9cm,yshift=9.1cm,scale=6,right] {\fantaisie 从};
\draw (cp)  node [xshift=-6cm,yshift=9.8cm,scale=3,right] {\fantaisie 入门};
\draw (cp)  node [xshift=-3.8cm,yshift=9.1cm,scale=6,right] {\fantaisie 到};
\draw (cp)  node [xshift=-6cm,yshift=8.6cm,scale=3,right] {\fantaisie 入土};

% Auteurs au centre sous le titre
\draw (cp)  node [xshift=-10cm,yshift=-10cm,scale=4,right,color={red},align=left] {十三飞};
%\draw (cp)  node [xshift=-8.7cm,yshift=6.95cm,scale=2,right,color={purple},align=left] {豆};
%\draw (cp)  node [xshift=-8.23cm,yshift=7.3cm,scale=3,right,color={blue},align=left] {三};
%\draw (cp)  node [xshift=-8.1cm,yshift=7.3cm,scale=4,right,color={violet},align=left] {飞};

% post'it en bas à droite avec les auteurs
\coordinate[xshift=4.5cm,yshift=-9.5cm] (postit) at (cp);
\postIt{(postit)}{17}{1.75} {\fantaisie
stay hungry}{}{\fantaisie stay foolish}
%L\raisebox{0.25\height}{A}T\raisebox{-0.25\height}{E}X et TikZ}


\end{tikzpicture}

\vfill
\parindent4pt